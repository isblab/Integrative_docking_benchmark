\section{Main Methods}

\subsection{sampleTheNode}
The exploration of the atlas is done by the \textbf{sampleTheNode} algorithm.
The input to this algorithm is an atlas node. Let $G_H$ be the active constraint
graph of the atlas Node and $H$ be the set of active constraints.
This algorithm recursively samples the atlas node till all its  descendants
have been sampled. The algorithm proceeds as follows
If active constraint graph $G_H$ of the root node is minimally rigid, we have
no more sampling to do, return. This forms the base case of the recursion.
If $G_H$ is not minimally rigid, find the set of parameters $F$ so that 
adding $F$ to $H$ makes the graph minimally rigid. Find the convex chart for
the parameters $F$. 
Next we have two nested for loops. The outer for loop runs for each Cayley 
point $p$ in the convex chart and computes the realization for each of these
points. The inner for loop runs for each realization $r$ of the point $p$. 
Though some Cayley points are realizable, they may violate some of the
constraints set by the user input. So, we do a constraint check to see
if anything is violated. If it is, we add the point to the violated set
otherwise, we add the realization to the point.	Then we check
if there is a witness point using the \textbf{boundaryDetection} algorithm.
During the boundary detection, if a new constraint $e$ becomes active,
we add create a new active constraint graph $G'(H)$ by adding $e$ to the
set $H$. If $G'_H$ hasn't already been sampled, we recursively call
sampleTheNode on $G'_H$.

\begin{algorithm} [htbp]
 \SetKwInOut{Input}{input}\SetKwInOut{Output}{output}

 {\bf sampleExplore}\\
 \Input{atlasNode}
 \Output{Complete sampling of the atlasNode and all its children}
 \BlankLine

	let $G_H$ be the activeConstraintGraph of the atlasNode with $H$ being active constraints\\
	\If{ $G_H$ is minimally rigid}
		{stop;	}
	//This needs another pseudocode\\
	find the parameters $F$ so that $G = (V, E = \{F\cup H\})$ is minimally rigid, i.e., complete 3-tree\\
	
	//This probably needs a pseudocode\\
	compute the convexChart for the parameters $F$\\

	\For{ each cayleyPoint $p$ within convexChart }
	{
		//This definitely needs a pseudocode\\
		compute all the realizations of $p$ for each flip\\

		\For{ each realization $r$ }
		{
			violated = ConstraintCheck($r$)\\
			\If{!violated}
			{

				add $o$ to $p$;\\
				get witness to boundary region if available through boundaryDetection(p);\\
				\If{ a new constraint $e$ becomes active in witness point from collection C }
				{
					$G'$ := $G_H \cup \{e\}$ ;\\
					\If{ $G'$ is not already present in the current atlas}
					{
						create child atlasNode labeled by $G'$\\
						save the child atlasNode to atlas\\
						sampleExplore(child);\\
					}
				}
			}
		}

		save $p$ to activeConstraintRegion of the activeConstraintGraph\\
	}

	\caption{sampleExplore}
\label{alg:sampleExplore}
\end{algorithm}




\subsection{boundaryDetection}

The \textbf{boundaryDetection} algorithm takes as input a Cayley point and finds the boundary using binary search. For each neighboring point $p_n$ of the input Cayley point $p$, we halve the step size run a while loop till a new constraint becomes active. Within the while loop, we again halve the step size and check if the witness point violates any steric constraints. If it does, we walk through p with the new step size, else we walk through $p_n$ with the new step size.


\begin{algorithm} [htbp]
 \SetKwInOut{Input}{input}\SetKwInOut{Output}{output}
 {\bf boundaryDetection}\\
 \Input{a Cayley point}
 \Output{a witness point}
 \BlankLine

	\For{ each neighboring point $p_n$ around p }
	{
		\If{ $p_n$ violates sterics}
		{
			// find boundary region in between $p$ and $p_n$ using binary search \\

			step\_size = step\_size/2;\\
			witness = walk back through $p$ with step\_size ;\\

			\While{ no new constraints in $G$ have become active }
			{
				step\_size = step\_size/2; 	\\
				\eIf{ witness is violated by sterics }
				{	witness = walk through $p$ with step\_size }
				{	witness = walk through $p_n$ with step\_size }
			}
		}
	}
	return witness;\\

\caption{ boundaryDetection
\label{alg:boundaryDetection}}
\end{algorithm}


\subsection{computeRealization}

The \textbf{computeRelization} algorithm, takes in an active constraint graph and its convex chart and generates all possible realizations for it. The active constraint graph can have eight flips we run a for loop for each of these eight flips. For each flip, we first check if the active constraint graph is a partial 3-tree. If it is, the computation of the realization is straightforward. We simply get the base tetrahedron and keep adding vertices to it till we get a complete 3-tree. If however the active constraint graph does not form a partial 3-tree, we drop a constraint and do ray tracing in one dimension higher.

\begin{algorithm}
\label{realization}
\SetKwInOut{Input}{input}\SetKwInOut{Output}{output}

 {\bf computeRealization}\\
 \Input{The active constraint graph, the convex chart}
 \Output{All possible realizations}
 \BlankLine
 	\For{each of the eight possible flips}
 	{
 		\If{partial 3 tree}{
 			Get the base tetrahedron\\
 			\While{The tetrahedron is not complete}{
	 			Add a new vertex edge-connected to the face of a tetrahedron.\\
	 			compute the Lengths of the edges connected to the new 	vertex\\
 			}
 		}
 		\Else{
 			Drop one constraint and do ray tracing in one dimension higher.
 		}
 	}

 \caption{computeRealization}
\end{algorithm}



\subsection{A Posteriori Boundary Detection}
This implements the a posteriori boundary detection discussed earlier.
It is called as a subroutine from the `sampleExplore' method. In
order to detect the (lower-dimensional) boundaries that always lie between a
feasible region and a violated region, for each feasible Cayley point we check
all Cayley parameter directions to see if the neighboring point is infeasible
or not. If it is, then binary search will take place to find out the
boundary point with the new active constraints. A boolean map of the grid that stores
the statuses (feasible/violated/infeasible) of all neighboring points, would
eliminate recreating neighbor points. However the size of the map would be
prohibitive for nodes of higher dimension.

\begin{algorithm} [htbp]
 \SetKwInOut{Input}{input}\SetKwInOut{Output}{output}

 {\bf boundaryDetection}\\
 \Input{cayleyPoint}
 \Output{cayleyPoint}
 \BlankLine

	\For{ each neighboring point $p_n$ around p }
	{
		\If{ $p_n$ is violated by sterics}
		{
			// find boundary region in between $p$ and $p_n$ using binary search \\

			step\_size = step\_size/2;\\
			witness = walk back through $p$ with step\_size ;\\

			\While{ no new constraints in $G$ have become active }
			{
				step\_size = step\_size/2; 	\\
				\eIf{ witness is violated by sterics }
				{	witness = walk through $p$ with step\_size }
				{	witness = walk through $p_n$ with step\_size }
			}
		}
	}
	return witness;\\

 \caption{ boundaryDetection
\label{alg:boundaryDetection}}
\end{algorithm}

%--------------
%(SpaceView -> ChartView)
%is it good to keep the solver and data structure in same class, as in ACG or cayleyparametrization?
%-------------

